%------------------------------------------------------------
% hw01.tex 
%
% This is a LaTeX2e document.
% Compile it twice to assure that all references are resolved.
% There is no .bib file associated with this document.
% 
% Remember: to quickly find what must be inserted at the last
% minute, search for lines with %%%include.
%------------------------------------------------------------
%
%**start of header
\documentclass{article}
\usepackage{epsfig}
\usepackage[myheadings]{fullpage}
\markboth{William J. De Meo}{Statistics 243}
%**end of header
\begin{document}
%------------------------------------------------------------
% REMEMBER TO CHANGE THE TITLE & DATE
%------------------------------------------------------------
\title{
\begin{tabbing}
\=Homework 1: \=\kill 
\>Homework 1: \>\emph{Function Prototypes,}\\
\>\>\emph{Machine Constants,}\\
\>\>\emph{and $\chi^2$ Critical Values}
\end{tabbing}
}
\author{William J. De Meo}
\date{October 7, 1997}
\maketitle

%------------------------------------------------------------
% ABSTRACT: ``Have the Summary clear and factual''
%------------------------------------------------------------
\begin{abstract}
This paper presents programs which exhibit three fundamental 
programming topics.  First we consider how function prototyping 
helps to guard against passing variables of different data types 
to a function.  Then we evaluate 
machine constants, such as the largest value of each data type 
and the smallest floating point number in single and double precision.
Finally, we construct
a program which supplies the $\chi^2$ critical value corresponding
to user specified degrees of freedom and probability level.
\end{abstract}

%------------------------------------------------------------
% INTRODUCTION: ``Set out the question; that is the scientific
%  question(s) driving the work.  Describe the data available''
%------------------------------------------------------------
\section{Introduction}
It is instructive to observe by example that specifying the data
type of a function's arguments when supplying a prototype for the
function will protect against the case when a variable of incorrect
data type is passed to that function.  In section~\ref{sec:sum},
we construct such an example to illustrate the different results 
one obtains when employing different function prototypes.

In many programming contexts, knowledge of various machine constants
is crucial.  These constants are required when deciding 
whether a particular problem is ill-conditioned with respect to 
your machine's capabilities, as well as whether a particular 
algorithm to solve the problem will execute stably on your machine.
For example, to address the latter concern, we might compare the
error in the computed solution to the \emph{machine epsilon}.  We
define machine epsilon as the smallest number such that, when added to 1, 
the result is greater than 1.  Intuitively, we think of this as the
smallest non-zero number.  Since this corresponds to the accuracy
with which the machine can represent a small error, we can't expect 
the error in our algorithm to be much smaller than machine epsilon.
On the other hand, an error much larger than machine epsilon,
calls the algorithm's stability into question.

Other numbers which are often required when deciding whether a problem
can be stably solved on your machine are the largest integer 
(both long and short) which can be handled by the machine, 
the largest unsigned integer, and the largest floating point number 
(both single and double precision).  In section~\ref{sec:mconstants}, 
we will construct a program which estimates each of these, give
results for the Sun Ultra 2, and demonstrate partial failure of the 
program on an Intel Pentium processor.\footnote{using the gcc compiler
under the Linux operating system.}

In statistical applications, it is often necessary to produce a 
critical value for a given probability distribution at a given 
level of significance.  In section~\ref{sec:chi}, we present a 
program which provides the critical value for the $\chi^2$ 
distribution with user specified degrees of freedom and level 
of significance.

%------------------------------------------------------------
% RESULTS: ``Provide the findings (e.g. the final models,
%  estimates, s.e.'s, c.i.'s, principal summarizing quantities,
%  (numerical & graphical)) and especially important points
%  (unexpected predictor variables,...).  Describe examinations
%  carried out of the basic assumptions made.''
%------------------------------------------------------------

\section{Programs}
\label{sec:Programs}

\subsection{Function Prototypes}
\label{sec:sum}
The program listing demonstrating the use of two different
prototypes appears in the appendix.  Some output from 
running this program appears below.\\\\
First trial:\\

{\tt
\% Sum 

No valid argument passed to Sum.\\

Enter first integer: 2\\

Enter second integer: 5\\

The sum of 2 and 5 is -2147483648.\\
}
\\
Second trial:\\

{\tt
\% Sum

No valid argument passed to Sum.\\
 
Enter first integer: 3\\
 
Enter second integer: 2\\
 
The sum of 3 and 2 is -2147483648.\\
}
\\
Third trial:\\

{\tt
\% Sum -s

Argument 1: -s

Entering Smart Mode...\\
 
Enter first integer: 2\\
 
Enter second integer: 5\\
 
The sum of 2 and 5 is 7.\\
}
\\
Fourth trial:\\

{\tt
\% Sum s

Argument 1: s

Entering Smart Mode...\\
 
Enter first integer: 3\\
 
Enter second integer: 2\\
 
The sum of 3 and 2 is 5.\\
}
\\
As the output makes clear, when the program is invoked with no option,
the answer is wrong.  The function prototype being used in this case
is\\

{\tt double sum();}\\
\\
When the {\tt -s} options is used, the prototype is\\

{\tt double sum(double, double)}\\
\\
which has the effect of automatically casting the integer arguments 
to double before evaluating the function.  Thus we get the correct
results.

\subsection{Machine Constants}
\label{sec:mconstants}
The listing for the program mconstants.c appears in the appendix, along
with it's output immediately following.  To summarize that output, we
see that the single precision machine epsilon is about 6.27583e-08.
The double precision machine epsilon is about 1.15829e-16.  The
largest short and long are 16384 and 1073741824, respectively.
The largest unsigned short and unsigned long are 32768, 2147483648,
respectively.  Finally, the largest single and double precision float
are 1.70141e+38 and 8.98847e+307, respectively.

\subsection{$\chi^2$ Critical Values}
\label{sec:chi}
A program which computes the critical values of the chi-square
distribution, with a user specified degrees of freedom and 
probability level, is listed in the appendix.  Some sample 
output appears below.\\
\\
{\tt 
\% chisquare\\
 
Enter the left tail probability level: .2\\
 
Enter the degrees of freedom: 15\\
 
The corresponding chi-square value is: 10.306963\\
 
That is, P( Chi-square\_(15) < 10.306963 ) = 0.200000\\
 
Press X to exit, or any other character to continue: y\\
 
Enter the left tail probability level: .3\\
 
Enter the degrees of freedom: 15\\
 
The corresponding chi-square value is: 11.721176\\
 
That is, P( Chi-square\_(15) < 11.721176 ) = 0.300000\\
 
Press X to exit, or any other character to continue: y\\
 
Enter the left tail probability level: .5\\
 
Enter the degrees of freedom: 10\\
 
The corresponding chi-square value is: 9.341823\\
 
That is, P( Chi-square\_(10) < 9.341823 ) = 0.500000\\
 
Press X to exit, or any other character to continue: X\\
\\
}
%------------------------------------------------------------
% DISCUSSION AND CONCLUSIONS:  ``Set out model and design
%  limitations, suggestions for future work, review the important
%  parts of the analyses.  Provide subject matter interpretation
%  of the conclusions.''
%------------------------------------------------------------
\section{Discussion and Conclusions}
The programs appearing in the appendix seem to give reasonable
and useful output, and answer the questions we posed to begin
this paper.  In particular, we have seen how a 
function fails when, expecting arguments of data type {\tt double},
it instead receives arguments of data type {\tt int}.  We have 
also seen how a more precise function prototype can guard against
this type of failure.  The {\tt mconstants.c} showed us the 
important machine constants for the Sun Ultra 2.  Finally, 
the {\tt chisquare.c} program computed critical values from
the $\chi^2$ distribution.
 
%------------------------------------------------------------
% APPENDIX:  Put important computer output here.
%------------------------------------------------------------
\section{Appendix}

{\bf Contents}
\begin{itemize}
\item Program listing for: {\tt sum.c}
\item Program listing for: {\tt mconstants.c}
\item Program output of: {\tt mconstants.c}
\item Program listing for: {\tt chisquare.c}
\end{itemize}
\subsection{Program listing for: {\tt sum.c}}
\begin{verbatim}

/*----------------------------------------------------------------------
  sum.c

  Created by William J. De Meo
    on 9/27/97

  This program adds two double precision floating point numbers.
  and was written to see what happens when such a function
  receives integer arguments.  

  The "dumb" version of the function is the default.  It expects double,
  and only double, precision floating point numbers.  If the -s option
  is specified, the "smart" version of the function is called, which gives
  the correct sum for integers, single floats, double floats, etc.
----------------------------------------------------------------------*/
#include <string.h>

main(int argc, char *argv[])	/* Possible arguments:        */
{				/* -s or s to invoke smart version */
     double sum();
     double ssum(double, double);

     int x,y,sflag=0;

     if(argv[1])
     {                               /* strcmp() returns zero if strings are equal */
	  if(!strcmp(argv[1],"-s") || !strcmp(argv[1],"s"))	
	  {
	       sflag=1;
	       printf("Argument 1: %s\nEntering Smart Mode...\n",argv[1]);
	  }
     }
     else printf("No valid argument passed to %s.\n",argv[0]);

     printf("\nEnter first integer: "); scanf("%d",&x);
     printf("\nEnter second integer: ");scanf("%d",&y);

     if(sflag) printf("\nThe sum of %d and %d is %d.\n",x,y,(int)ssum(x,y));
     else       printf("\nThe sum of %d and %d is %d.\n",x,y,(int)sum(x,y));
}

double sum(double x, double y)
{
     return(x+y);
}

double ssum(double x, double y) {sum(x,y);} 
/* ssum() gets the same function definition */
\end{verbatim}

\subsection{Program listing for: {\tt mconstants.c}}
\begin{verbatim}
/*----------------------------------------------------------------------
  mconstants.c

  Created by William J. De Meo
    on 9/27/97

  This program's purpose is to find various machine parameters.
  (All numbers have been type casted in an attempt to prevent the compiler 
  from forcing higher precision.)
----------------------------------------------------------------------*/
#include <stdlib.h>

main()
{

  short S, bigS;
  long L, bigL;
  unsigned short US, bigUS;
  unsigned long UL, bigUL;
  float F, bigF;
  double D, bigD;
  float delta, eps, negeps;
  double ddelta, deps, dnegeps;


  /* First let's check out the size of each data type on this machine */
  printf("\n      short: %d bytes.",sizeof(short));
  printf("\n        int: %d bytes.",sizeof(int));
  printf("\n       long: %d bytes.",sizeof(long));
  printf("\n      float: %d bytes.",sizeof(float));
  printf("\n     double: %d bytes.",sizeof(double));
  printf("\nlong double: %d bytes.",sizeof(long double));
printf("\n----------------------------------------------------------------------\n");

  /* Find Largest Short */
  S = (short) 1;
  do
    {
      bigS = S;
      S *= (short) 2;
    }while(bigS == (S / (short) 2));   /* If S*2 too big for machine, then  (S*2)/2 != S */
                            /* and loop exits.  S stores the spurious (oversized) value.  */

  /* Find Largest Long */
  L = (long) 1;
  do
    {
      bigL = L;
      L *= (long) 2;
    }while(bigL == (L / (long) 2)); 


  /* Find Largest Unsigned Short */
  US = (unsigned short) 1;
  do
    {
      bigUS = US;
      US *= (unsigned short) 2;
    }while(bigUS == (US / (unsigned short) 2)); 
                                  
  /* Find Largest Unsigned Long */
  UL = (unsigned long) 1;
  do
    {
      bigUL = UL;
      UL *= (unsigned long) 2;
    }while(bigUL == (UL / (unsigned long) 2)); 

  /* Find Largest Single Precision Float */
  F = (float) 1;
  do
    {
      bigF = F;
      F *= (float) 2;
    }while(bigF == (F / (float) 2)); 

  /* Find Largest Double Precision Float */
  D = (double) 1;
  do
    {
      bigD = D;
      D *= (double) 2;
    }while(bigD == (D / (double) 2)); 


  /*** MACHINE EPSILON ***/

  /* Find Smallest Single Precision Float */
  /* Two methods produce different results */
  /* The results are related by:  eps = 2 * negeps  */


  /* First method: test that addition to 1 is greater than 1 */
for(delta= (float) 1 ;(float) 1 + delta > (float) 1;)
    {
      eps=delta;
      delta /= (float) 1.1;		
    }

  /* Second method: test that subtraction from 1 is less than 1 */

  for(delta= (float) 1;(float)((float) 1 - delta) < (float) 1;)
    {
      negeps=delta;
      delta /= (float) 1.1;
    }

  /* Find Smallest Double Precision Float (agian by two methods) */
  ddelta = (double) 1;
  while((double) 1 - ddelta < (double) 1)
    {
      dnegeps=ddelta;
      ddelta /= (double) 1.1;		
    }

  for(ddelta = (double) 1; (double) 1 + ddelta > (double) 1; )
    {
      deps=ddelta;
      ddelta /= (double) 1.1;		
    }

  printf("\n                      largest short and long:  %d, %d",bigS, bigL);
  printf("\n    largest unsinged short and unsigned long:  %u, %u",bigUS, bigUL);
  printf("\n   largest single and double precision float:  %g, %g",bigF, bigD);
  printf("\nsingle precision machine epsilon and neg-eps:  %g, %g",eps, negeps);
  printf("\ndouble precision machine epsilon and neg-eps:  %g, %g",deps,dnegeps);
  printf("\n----------------------------------------------------------------------\n");
  printf("\nValues of Variables After Overflow");
  printf("\n----------------------------------------------------------------------\n");
  printf("                     short and long:  %d, %d",S,L);
  printf("\n unsinged short and unsigned long:  %u, %u",US, UL);
  printf("\nsingle and double precision float:  %g, %g",F, D);
  printf("\n----------------------------------------------------------------------\n");
  printf("\nValues of Variables After Underflow");
  printf("\n----------------------------------------------------------------------\n");
  printf("  single precision machine eps:  %g",delta);
  printf("\ndouble precision machine eps:  %g",ddelta);
  printf("\n----------------------------------------------------------------------\n");

printf("\n\nProof that computed epsilons are not zero:");
  printf("\n----------------------------------------------------------------------\n");
printf("\nSingle precision (25 decimal display):");
printf("\n   1 + eps = %.25e",(double) 1 + eps);
printf("\n1 - negeps = %.25e\n", (double) 1 - negeps);
printf("\n\nDouble precision (25 decimal display):");
printf("\n   1 + eps = %.25e",(double) 1 + deps);
printf("\n1 - negeps = %.25e\n", (double) 1 - dnegeps);
  printf("\n----------------------------------------------------------------------\n");
}
\end{verbatim}

\subsection{Program output of: {\tt mconstants.c}}
\begin{verbatim}
  /********************************************/
 /**  Output from the mconstants.c program  **/
/********************************************/


      short: 2 bytes.
        int: 4 bytes.
       long: 4 bytes.
      float: 4 bytes.
     double: 8 bytes.
long double: 16 bytes.
----------------------------------------------------------------------

                      largest short and long:  16384, 1073741824
    largest unsinged short and unsigned long:  32768, 2147483648
   largest single and double precision float:  1.70141e+38, 8.98847e+307
single precision machine epsilon and neg-eps:  6.27583e-08, 3.22049e-08
double precision machine epsilon and neg-eps:  1.15829e-16, 5.94384e-17
----------------------------------------------------------------------

Values of Variables After Overflow
----------------------------------------------------------------------
                     short and long:  -32768, -2147483648
 unsinged short and unsigned long:  0, 0
single and double precision float:  Inf, Inf
----------------------------------------------------------------------

Values of Variables After Underflow
----------------------------------------------------------------------
  single precision machine eps:  2.92772e-08
double precision machine eps:  1.05299e-16
----------------------------------------------------------------------


Proof that computed epsilons are not zero:
----------------------------------------------------------------------

Single precision (25 decimal display):
   1 + eps = 1.0000000627583176537882537e+00
1 - negeps = 9.9999996779506261646019993e-01


Double precision (25 decimal display):
   1 + eps = 1.0000000000000002220446049e+00
1 - negeps = 9.9999999999999988897769754e-01

----------------------------------------------------------------------

\end{verbatim}

\subsection{Program listing for: {\tt chisquare.c}}
\begin{verbatim}
/*----------------------------------------------------------------------
  chisquare.c

  Created by William J. De Meo
    on 10/7/97

    Purpose:  This program obtains degrees of freedom (df) and 
    left tail probability level (p) from the user, 
    generates the corresponding Chi-square value, and awaits another query.

    Notes: On the SCF network, type xnaghelp for documentation regarding 
           the NAG subroutine g01fcf_
----------------------------------------------------------------------*/
#include <stdlib.h>
#include <stdio.h>

void main()
{
     double df,p,chi;
     char another='y';
     int ifail;			/* -1 instructs routine to */
     /* report errors and continue */

     double g01fcf_(double *p, double *df, int *ifail);

     while(((another!='x') && (another!='X')) || another=='\0')
     {
	  printf("\nEnter the left tail probability level: "); scanf("%lf",&p);
	  printf("\nEnter the degrees of freedom: "); scanf("%lf",&df);

	  ifail = -1;
	  chi = g01fcf_(&p,&df,&ifail);

	  if(ifail != 0)
	  {
	       switch(ifail){
	       case 1:
		    printf("\nProbability level out of range [0,1).");
		    break;
	       case 2:
		    printf("\nDegrees of freedom not greater than 0.");
		    break;
	       case 3:
		    printf("\nProbability too close to 0 or 1 for value to be calculated.");
		    break;
	       case 4:
		    printf("\nSolution failed to converge. Result is only an approximation.");
		    printf("\nThe corresponding chi-square value is: %lf\n", chi);
		    printf("\nThat is, P( Chi-square_(%lf) < %lf ) = %lf", df,chi,p);

		    break;
	       case 5:
		    printf("\nSolution failed to converge.");
		    break;
	       default:
		    break;
	       }
	  }
	  else
	  {
	       printf("\nThe corresponding chi-square value is: %lf\n", chi);
	       printf("\nThat is, P( Chi-square_(%d) < %lf ) = %lf", (int)df,chi,p);
	  }
	  printf("\n\nPress X to exit, or any other character to continue: ");
	  scanf("%1s",&another);
     }
}

\end{verbatim}
\end{document}

