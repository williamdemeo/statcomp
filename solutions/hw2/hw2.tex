%------------------------------------------------------------
% Template .tex file for Stat 243 labs
%
% This is a LaTeX2e document.
% Compile it twice to assure that all references are resolved.
% There is no .bib file associated with this document.
%------------------------------------------------------------
%
%**start of header
\documentclass{article}
\usepackage[myheadings]{fullpage}
\markboth{William J. De Meo}{Statistics 243}
\usepackage{epsfig}
\newcommand{\mod}{\mbox{mod }}
%**end of header
\begin{document}
%------------------------------------------------------------
% REMEMBER TO CHANGE THE TITLE & DATE
%------------------------------------------------------------
\title{Statistics 243: \emph{homework 2}}
\author{William J. De Meo}
\date{November 2, 1997}
\maketitle

%------------------------------------------------------------
% METHODS: ``Indicate the important analytic tools to be
%  employed, with associated assumptions and an indication of why.''
%------------------------------------------------------------
\section{Mean and Variance Algorithms}
We construct three algorithms -- desk calculator, provisional
means, and centered about first observation -- for computing the 
mean and variance of a list of numbers.  The C source file {\tt
  moment.c}, whose listing is given in the Appendix in
section~\ref{sec:moment}, contains routines which carry out each of 
these algorithms.  The output of the program {\tt illtest.c}, which
calls the three routines, appears in section~\ref{sec:illout}.  As 
is clear from this output, we first detect relative
errors of size 0.1 in the desk calculator algorithm when 
the coefficient of variation is near $10^{-8}$, which is about
the square root of machine epsilon.  The provisional means 
algorithm and the centering algorithm appear to do fine until 
the coefficient of variation is of order machine epsilon.

\section{Uniform Random Numbers}
The multiplicative congruential generator, 
\[
X_{n+1} = aX_{n} \;\;(\mod m)
\]
has been shown to work well both theoretically and empirically if
the multiplier $a$ and the modulus $m$ are chosen carefully.
The modulus $2^{32}$ and multiplier $a$ such that $a \mod 8$
is 3 or 5 have been proposed.  In order to implement this while
avoiding overflow, we can use Schrage's~\cite{schrage} 
\emph{approximate factorization algorithm.}  Factorize $m$ as
\begin{equation}
\label{eqn:factor}
m = aq + r
\end{equation}
That is, $q$ is the integer part of $m/a$ while $r = m \mod a$.
If $r<q$ and $0 < X_n < m-1$, it can be shown that 
\begin{equation}
\label{eqn:schrage}
aX_n \mod m = \left\{ 
\begin{array}{l r}
a(X_n \mod q) - r [X_n/q] & \mbox{if it is }\geq 0\\
a(X_n \mod q) - r [X_n/q] + m & \mbox{otherwise}
\end{array}\right.
\end{equation}
where $[x]$ represents the largest integer no greater than $x$.
If we put $m = 2^{32}$ and $a = 16805$ (which satisfies the requirement
$a \mod 8 = 5$), then, performing a little paper and pencil
algebra, we can see that the values $q = 255576$ and $r = 12614$
do the trick.  Nonetheless, in order to implement the algorithm
for $m = 2^{32}$, we still need to access the value $2^{32}$ 
which will cause overflow on many platforms.  
The more portable suggestion, described
in Park and Miller~\cite{P&M}, is called the ``Minimal Standard''
generator, and is based on the choices,
\[a = 7^5 = 16807 \;\; m = 2^{31} - 1 = 2147483647 \]
Applying Schrage's approximation to these values gives $q = 127773$
and $ r = 2836$.  We use these values to implement the 
version of the multiplicative
congruential generator given by (\ref{eqn:schrage}).
The C program and results appear in section~\ref{sec:unif}.
As we expect, the output shows that the average of the pseudo-random
uniform(0,1) random numbers is near 1/2, and the variance is
near $1/12 = 0.08\bar{3}$.

\section{Normal Random Numbers}
We construct normal random numbers using the polar method.  
Two random normal(0,1) numbers, $x_1,x_2$, are constructed from two 
uniform(0,1) numbers, $u_1,u_2$ (e.g. the kind described above) 
using the following steps:
\begin{enumerate}
\item generate $u_1, u_2 \sim$ uniform(0,1)
\item $v_j = 2u_j - 1,\;\; j=1,2$ so that $v_j \sim$ uniform(-1,1)
\item $s = v_1^2 + v_2^2$
\item if $s\geq 1$ go back to first step and generate new $u_1,u_2$
\item $x_1 = v_1 \sqrt{\frac{-2 \log s}{s}}$ and 
$x_2 = v_2 \sqrt{\frac{-2 \log s}{s}}$ 
\end{enumerate}
This results in $x_j \sim$ normal(0,1), $j=1,2$.  The probability
that we will be forced to generate new uniforms (i.e. return to the first
step in the algorithm) is the probability that $v_1^2 +v_2^2 < 1$ which,
since $v_j$ are uniform(-1,1), is just $\pi/4$.  Therefore, using the negative
binomial distribution, it is easy to see that the expected number of rejections
is $nq/p \approx 0.27n$.  When we implement the polar method in the program
{\tt rand.c} (section~\ref{sec:rand}), we ask the user how many normal random
numbers are required, then the program generates twice as many uniform 
random numbers, from which an attempt is made at getting the required number
of normals. 

As can be seen from the output in section~\ref{sec:randout}, 
the means and variances (we expect around
0 and 1) look reasonable.  Furthermore, we expect only about five percent
of the observation to be more than 1.96 in absolute value.  This also appears
to be the case for our samples.

%------------------------------------------------------------
% APPENDIX:  Put important computer output here.
%------------------------------------------------------------
\section{Appendix}
\subsection{source code for illtest.c}
{\tt
\begin{verbatim}
/*~~~~~~~~~~~~~~~~~~~~~~~~~~~~~~~~~~~~~~~~~~~~~~~~~~~~~~~~~~~~
   illtest.c

   purpose:  Test the routines dmoment, pmoment, cmoment in 
             file moment.c on illconditioned data .
~~~~~~~~~~~~~~~~~~~~~~~~~~~~~~~~~~~~~~~~~~~~~~~~~~~~~~~~~~~~*/
#include <stdio.h>
#include <stdlib.h>
#include <math.h>
#include "prototypes.h"

#define small 0.0000001
#define BIG 10

main()
{
     double  *relerr, *dave, *dvar, *pave, *pvar, *cave, *cvar, *x;
     double VAR, coef;
     unsigned long i=0,n, count=0;
     int flag=0, j;
     FILE *ofp;       
  
     dave = dmalloc(1);
     dvar = dmalloc(1);
     pave = dmalloc(1);
     pvar = dmalloc(1);
     cave = dmalloc(1);
     cvar = dmalloc(1);
  
     printf("\nHow many data values? ");
     scanf("%d",&n);
     x = dmalloc(n);

     printf("\nEnter data values at the > prompt. ");
     for(i=0;i<n;i++){
          printf("\n> ");
          scanf("%lf",&x[i]);
     }
     dmoment(x, n, dave, dvar);
     pmoment(x, n, pave, pvar);
     cmoment(x, n, cave, cvar);

     ofp = fopen("illout", "w");
     fprintf(ofp,"\nx = \n");
     for(i=0; i<n; i++) fprintf(ofp," %lf ",x[i]);
     fprintf(ofp,"\n\n(dave,dvar) = (%lf,%lf)",*dave,*dvar);
     fprintf(ofp,"\n\n(pave,pvar) = (%lf,%lf)",*pave,*pvar);
     fprintf(ofp,"\n\n(cave,cvar) = (%lf,%lf)\n",*cave,*cvar);

     /* Ill-conditioning */

     VAR = *dvar; /* the "true" mean and variance */
     relerr = dmalloc(3);

     while(flag<2) 
     {
          count++;          
          for(i=0;i<n;i++) x[i]+=pow(2,count); 

          dmoment(x, n, dave, dvar);
          pmoment(x, n, pave, pvar);
          cmoment(x, n, cave, cvar);

          relerr[0] = (*dvar - VAR)/VAR;
          relerr[1] = (*pvar - VAR)/VAR;
          relerr[2] = (*pvar - VAR)/VAR;
          coef = sqrt(VAR)/(*pave);
                        
          /* If any relative errors are bigger than small, */
          /* print a warning.  If any are bigger than BIG, exit. */
          for(j=0;j<3;j++) {
               if(relerr[j] > small || -relerr[j] > small) 
                  fprintf(ofp,
                    "\nWarning: on iteration %u, relerr[%d] = %g, coef = %g",
                    count,j,relerr[j],coef);
               if(relerr[j] > BIG || -relerr[j] > BIG) {
                  fprintf(ofp,
                    "\nERROR:   on iteration %u, relerr[%d] = %g, coef = %g",
                    count, j,relerr[j],coef);
                  flag++;
               }
          }
     }
     fprintf(ofp,"\n\n%s%u%s%s%lf %s%lf %s%lf %s%lf\n ",
             "On iteration ", count,
             " at least two gross errors:\n",
             "VAR = ", VAR,
             "\ndvar = ", *dvar, 
             "\npvar = ", *pvar, 
             "\ncvar = ", *cvar);
     fclose(ofp);
}
\end{verbatim}
}
\subsection{source code for moment.c}
\label{sec:moment}
{\tt
\begin{verbatim}
/*~~~~~~~~~~~~~~~~~~~~~~~~~~~~~~~~~~~~~~~~~~~~~~~~~~~~~~~~~~~~
   moment.c
  
   Three routines for for computing the mean and variance
~~~~~~~~~~~~~~~~~~~~~~~~~~~~~~~~~~~~~~~~~~~~~~~~~~~~~~~~~~~~*/
/* dmoment: The desk calculator algorithm
   arguments:
              data = a nx1 array of doubles
              n = length of data[]
              ave =(on entry)= the address of *ave (call by reference)
              ave =(on exit)= average of data[]
              var =(on entry)= the address of *var (call by reference)
              var =(on exit)= variance of data[]
              */
void dmoment(double * data, unsigned int n, double *ave, double *var)
{
     unsigned int i;
  
     *ave=0; *var=0;
  
     for(i=0;i<n;i++){
          *ave += data[i];
          *var += data[i]*data[i];
     }
     *ave /= (double) n;
     *var = (*var - (double)n*((*ave)*(*ave)))/(double)(n-1);
}

/* pmoment: The provisional means algorithm
   arguments:
              data = a nx1 array of doubles
              n = length of data[]
              ave =(on exit)= the average of data[]
              var =(on exit)= the variance of data[]
              */
void pmoment(double * data, unsigned int n, double *ave, double *var)
{
     unsigned int i;

     *ave = data[0];
     *var=0;
  
     for(i=1;i<n;i++){
          *var +=((double)i/(double)(i+1))*(data[i]-*ave)*(data[i]-*ave);
          *ave *= ((double)i/(double)(i+1));
          *ave += data[i]/(double)(i+1);
     }
     *var /= (double)(n-1);
}

/* cmoment: centering around the first observation 
   arguments:
              data = a nx1 array of doubles
              ave =(on exit)= the average of data[]
              var =(on exit)= the variance of data[]
              */
void cmoment(double * data, unsigned int n, double *ave, double *var)
{
     unsigned int i;
  
     *ave=0;
     *var=0;
     for(i=1;i<n;i++){
          *ave += data[i] - data[0];
          *var += (data[i] - data[0])*(data[i] - data[0]);
     }
     *ave /= (double)n;
     *var = (*var - (double)n * ((*ave)*(*ave)))/(double)(n-1);
     *ave += data[0];
}
\end{verbatim}
}

\subsection{makefile for illtest (and its precursor momtest)}
{\tt
\begin{verbatim}
# Makefile for illtest momtests

HOME = /home/chip
HOMEBASE = $(HOME)/classes/s243/programs/hw2
HOMELIB = $(HOME)/lib
INCL = -I$(HOMELIB)/include
CC = gcc

# Ill-conditioned data test
illtest: illtest.o moment.o $(HOMELIB)/util_lib.a
        $(CC) -o illtest -lm illtest.o moment.o $(HOMELIB)/util_lib.a 

illtest.o: illtest.c
        $(CC) $(INCL) -c illtest.c

# Ill-conditioned data test (debugging)
illtest.db: illtest.do moment.do $(HOMELIB)/util_lib.a
        $(CC) -o illtest.db -lm illtest.do moment.do $(HOMELIB)/util_lib.a

illtest.do: illtest.c
        $(CC) $(INCL) -o illtest.do -c -g illtest.c

# (moment.do: is made below)


# Ordinary routine test
momtest: momtest.o moment.o $(HOMELIB)/util_lib.a
        $(CC) -o momtest momtest.o moment.o $(HOMELIB)/util_lib.a

momtest.o: momtest.c
        $(CC) $(INCL) -c momtest.c

moment.o: moment.c
        $(CC) -c moment.c

# Ordinary routine test debugging
momtest.db: momtest.do moment.do $(HOMELIB)/util_lib.a
        $(CC) -o momtest.db momtest.do moment.do $(HOMELIB)/util_lib.a

momtest.do: momtest.c
        $(CC) $(INCL) -o momtest.do -c -g momtest.c

moment.do: moment.c
        $(CC) -o moment.do -c -g moment.c
\end{verbatim}
}

\subsection{output from illtest}
\label{sec:illout}
The following output was produced upon entering five consective
digits (shown in the {\tt x} vector below).
{\tt
\begin{verbatim}
x = 
 1.000000  2.000000  3.000000  4.000000  5.000000 

(dave,dvar) = (3.000000,2.500000)

(pave,pvar) = (3.000000,2.500000)

(cave,cvar) = (3.000000,2.500000)

Warning: on iteration 25, relerr[0] = 0.1, coef = 2.35608e-08
Warning: on iteration 26, relerr[0] = 0.1, coef = 1.17804e-08
Warning: on iteration 27, relerr[0] = -1.5, coef = 5.8902e-09
Warning: on iteration 28, relerr[0] = -1.5, coef = 2.9451e-09
Warning: on iteration 29, relerr[0] = -1.5, coef = 1.47255e-09
Warning: on iteration 30, relerr[0] = -1.4, coef = 7.36275e-10
Warning: on iteration 31, relerr[0] = -1, coef = 3.68138e-10
Warning: on iteration 32, relerr[0] = -1, coef = 1.84069e-10
Warning: on iteration 33, relerr[0] = -1, coef = 9.20344e-11
Warning: on iteration 34, relerr[0] = -1, coef = 4.60172e-11
Warning: on iteration 35, relerr[0] = -1, coef = 2.30086e-11
Warning: on iteration 36, relerr[0] = -1, coef = 1.15043e-11
Warning: on iteration 37, relerr[0] = -1, coef = 5.75215e-12
Warning: on iteration 38, relerr[0] = -1, coef = 2.87607e-12
Warning: on iteration 39, relerr[0] = -1, coef = 1.43804e-12
Warning: on iteration 40, relerr[0] = -1, coef = 7.19019e-13
Warning: on iteration 41, relerr[0] = -1, coef = 3.59509e-13
Warning: on iteration 42, relerr[0] = -1, coef = 1.79755e-13
Warning: on iteration 43, relerr[0] = -1, coef = 8.98773e-14
Warning: on iteration 44, relerr[0] = -1, coef = 4.49387e-14
Warning: on iteration 45, relerr[0] = -1, coef = 2.24693e-14
Warning: on iteration 46, relerr[0] = -1, coef = 1.12347e-14
Warning: on iteration 47, relerr[0] = -1, coef = 5.61733e-15
Warning: on iteration 48, relerr[0] = -1, coef = 2.80867e-15
Warning: on iteration 49, relerr[0] = -1, coef = 1.40433e-15
Warning: on iteration 50, relerr[0] = -1, coef = 7.02167e-16
Warning: on iteration 51, relerr[0] = -9.0072e+14, coef = 3.51083e-16
ERROR:   on iteration 51, relerr[0] = -9.0072e+14, coef = 3.51083e-16
Warning: on iteration 51, relerr[1] = 0.22, coef = 3.51083e-16
Warning: on iteration 51, relerr[2] = 0.22, coef = 3.51083e-16
Warning: on iteration 52, relerr[0] = 7.20576e+15, coef = 1.75542e-16
ERROR:   on iteration 52, relerr[0] = 7.20576e+15, coef = 1.75542e-16
Warning: on iteration 52, relerr[1] = 0.63, coef = 1.75542e-16
Warning: on iteration 52, relerr[2] = 0.63, coef = 1.75542e-16

On iteration 52 at least two gross errors:
VAR = 2.500000 
dvar = 18014398509481984.000000 
pvar = 4.075000 
cvar = 4.000000
\end{verbatim} 
}
\subsection{source code for unif.c}
\label{sec:unif}
{\tt
\begin{verbatim}
/*~~~~~~~~~~~~~~~~~~~~~~~~~~~~~~~~~~~~~~~~~~~~~~~~~~~~~~~~~~~~
  unif.c

  Generate uniform(0,1) random variables
~~~~~~~~~~~~~~~~~~~~~~~~~~~~~~~~~~~~~~~~~~~~~~~~~~~~~~~~~~~~*/

#include <math.h>
#include "prototypes.h"

#define A 16807 
#define M 2147483647 
#define Q 127773
#define R 12614

double unif(long *X);

double unif(long *X)
{
     long k;
     double unif;

     k = (*X)/Q;                   /* [x/q]                    */
     *X = A * (*X - k*Q) - R*k;    /* x - k*q = x mod q        */
     if(*X < 0) *X += M;
     unif = (*X)*((double)1.0/M);  /* make it uniform on (0,1) */
     return(unif);
}

main()
{
     unsigned long n, i;
     double *u, *ave, *var;
     long *X;

     ave = dmalloc(1);
     var = dmalloc(1);
     X = lmalloc(1);
     *X = time('\0');

     printf("\nHow many uniform random variables? ");
     scanf("%d",&n);
     u = dmalloc(n);

     printf("The unif(0,1) random variables are:\n\n");
     for(i=0;i<n;i++)
     { 
          u[i] = unif(X);
          printf("%lf  ",u[i]);
          if((i+1)%6 == 0) printf("\n");
     }
     cmoment(u, n, ave, var);
     printf("\nAverage = %lf, Variance = %lf\n",*ave,*var);
}
\end{verbatim}
}
\subsection{output from unif.c}
The following output file was produced by invoking the {\tt unif}
program three times, each time requesting 100 pseudo-random numbers.
{\tt
\begin{verbatim}
The unif(0,1) random variables are:

0.891198  0.293737  0.814619  0.236471  0.346485  0.339607  
0.756221  0.751822  0.808802  0.470374  0.539402  0.692317  
0.718020  0.707460  0.234207  0.291148  0.308513  0.156884  
0.735386  0.579936  0.947248  0.320143  0.622354  0.863478  
0.406654  0.599327  0.847265  0.920982  0.869615  0.549835  
0.032264  0.254038  0.594280  0.017005  0.804019  0.091891  
0.406409  0.488968  0.041630  0.675458  0.367903  0.312026  
0.194511  0.129265  0.540316  0.047268  0.421787  0.945138  
0.866346  0.605256  0.483753  0.406308  0.788354  0.799639  
0.464047  0.202129  0.170173  0.076436  0.659413  0.709490  
0.336908  0.385586  0.515348  0.411696  0.348228  0.635149  
0.894460  0.114777  0.048596  0.746734  0.301957  0.963401  
0.814759  0.589734  0.606227  0.816773  0.444151  0.814875  
0.535618  0.092193  0.483548  0.952049  0.008366  0.602685  
0.274745  0.609963  0.593704  0.333985  0.258171  0.060376  
0.742472  0.666400  0.129082  0.465106  0.007969  0.928584  
0.648206  0.353985  0.404499  0.388195  

Average = 0.495443, Variance = 0.074287

The unif(0,1) random variables are:

0.891949  0.921305  0.309861  0.812346  0.039243  0.546151  
0.113200  0.550127  0.947623  0.635354  0.348007  0.924881  
0.401476  0.582573  0.264982  0.540465  0.554694  0.701083  
0.054954  0.603580  0.315017  0.466426  0.188175  0.636627  
0.741894  0.959982  0.342052  0.846062  0.697293  0.351566  
0.735897  0.168858  0.988652  0.197509  0.518474  0.947082  
0.540708  0.636511  0.790586  0.318743  0.093083  0.432440  
0.980287  0.609730  0.690539  0.843903  0.419434  0.389202  
0.289199  0.541784  0.723724  0.574020  0.518557  0.355256  
0.764091  0.027336  0.434970  0.509588  0.609256  0.715255  
0.238892  0.040068  0.425652  0.899911  0.741362  0.008386  
0.943828  0.849288  0.921455  0.825286  0.522647  0.090202  
0.022260  0.119488  0.218426  0.065443  0.892779  0.865183  
0.066475  0.234757  0.545922  0.270902  0.024762  0.167835  
0.786539  0.294403  0.017080  0.069918  0.100711  0.638742  
0.291831  0.779917  0.002764  0.455749  0.736225  0.672166  
0.046296  0.090959  0.737707  0.590821  

Average = 0.489667, Variance = 0.088923

The unif(0,1) random variables are:

0.892286  0.577408  0.444576  0.963186  0.187000  0.893340  
0.294606  0.412780  0.557117  0.427264  0.993391  0.841828  
0.542357  0.347577  0.698716  0.273145  0.731618  0.242505  
0.758537  0.667052  0.093947  0.961029  0.935320  0.846348  
0.506152  0.851337  0.361794  0.642129  0.210016  0.719121  
0.217387  0.604308  0.558331  0.825282  0.447318  0.044756  
0.206730  0.503358  0.905271  0.820670  0.934329  0.204014  
0.855518  0.621555  0.422413  0.464824  0.263866  0.767518  
0.614276  0.095455  0.309792  0.658259  0.302246  0.833454  
0.803811  0.587230  0.526927  0.019076  0.607356  0.790122  
0.525837  0.699197  0.354821  0.453175  0.482786  0.139194  
0.425180  0.959356  0.821873  0.161693  0.566057  0.668648  
0.918881  0.570908  0.205213  0.004371  0.457523  0.560433  
0.147534  0.587341  0.400331  0.339136  0.834900  0.095730  
0.922746  0.519107  0.588993  0.154513  0.885645  0.965169  
0.517298  0.187958  0.993685  0.786826  0.131531  0.629180  
0.583683  0.911382  0.524533  0.790230  

Average = 0.551615, Variance = 0.074321

\end{verbatim}
}
\subsection{source code for random.c, normal.c, and uniform.c}
\label{sec:rand}
{\tt
\begin{verbatim}

/*~~~~~~~~~~~~~~~~~~~~~~~~~~~~~~~~~~~~~~~~~~~~~~~~~~~~~~~~~~~~
  random.c

  Generate normal(0,1) random variables
~~~~~~~~~~~~~~~~~~~~~~~~~~~~~~~~~~~~~~~~~~~~~~~~~~~~~~~~~~~~*/

#include <math.h>
#include <stdio.h>
#include "prototypes.h"

double max(double a, double b, double c);
double min(double a, double b, double c);

main()
{
     unsigned long n, m, i,j;
     double *u, *g, *gtemp, *ave, *var, minimum = 100, maximum = -100;
     long *X;
     FILE *ofp;

     ave = dmalloc((unsigned long)1);
     var = dmalloc((unsigned long)1);
     gtemp = dmalloc((unsigned long)2);
     X = lmalloc((unsigned long)1);
     *X = time('\0');

     printf("\nHow many normal random variables? ");
     scanf("%u",&m);
     n = 2*m; /* generate two times as many uniforms */

     ofp = fopen("norm.out", "a");

     u = dmalloc(n);
     for(i=0;i<n;i++)
          u[i] = unif(X);

     cmoment(u, n, ave, var);
     fprintf(ofp,"\n\nUAverage = %lf, UVariance = %lf\n",*ave,*var);

     fprintf(ofp,"\nThe normal(0,1) random variables are:\n\n");
     g = dmalloc(m);
     for(i=0;i<m;i++)
     {
          if(normal(u, gtemp,n)==1)
          {
               g[i] = gtemp[0]; 
               g[++i] = gtemp[1];
               maximum = max(maximum,g[i-1], g[i]);
               minimum = min(minimum,g[i-1], g[i]);
               fprintf(ofp,"%lf  %lf  ",g[i-1], g[i]); 
               if((i+1)%6 == 0) fprintf(ofp,"\n"); 
          }
          else  /* didn't get enough normals -- need new uniforms */
          {
               i = 0;
               printf("\n\nGenerating different unif(0,1) variables...\n\n");
               for(j=0;j<n;j++)
                    u[j] = unif(X);
          }
     }
     cmoment(g, m, ave, var);
     fprintf(ofp,"\n\nAverage = %lf, Variance = %lf, min = %lf, max = %lf \n"
             ,*ave,*var, minimum, maximum);
     fclose(ofp);
}

double max(double a, double b, double c)
{
     double max;
     if(a >= b) max = a;
     else max = b;

     if(max >= c) return(max);
     else return(c);
}

double min(double a, double b, double c)
{
     double min;
     if(a <= b) min = a;
     else min = b;

     if(min <= c) return(min);
     else return(c);
}

\end{verbatim}
}
{\tt
\begin{verbatim}
/*~~~~~~~~~~~~~~~~~~~~~~~~~~~~~~~~~~~~~~~~~~~~~~~~~~~~~~~~~~~~
   normal.c

     Routine for constructing normal(0,1) random numbers 
~~~~~~~~~~~~~~~~~~~~~~~~~~~~~~~~~~~~~~~~~~~~~~~~~~~~~~~~~~~~*/
#include <math.h>
#include "prototypes.h"

int normal(double *u, double *x, unsigned long n)
{
     double s;
     static int i = -2;

     do{
          i += 2;
          s = (2*u[i] - 1)*(2*u[i] - 1) + (2*u[i+1] - 1)*(2*u[i+1] - 1);
          if(i > n-2) 
          {
               i=-2;
               
               return(0);
          }
     }while(s >= 1);

     x[0] = (2*u[i] - 1)*sqrt(-2*log(s)/s); 
     x[1] = (2*u[i+1] - 1)*sqrt(-2*log(s)/s);

     return(1);

}
\end{verbatim}
}
{\tt
\begin{verbatim}
/*~~~~~~~~~~~~~~~~~~~~~~~~~~~~~~~~~~~~~~~~~~~~~~~~~~~~~~~~~~~~
  uniform.c

  Generate uniform(0,1) random variables
~~~~~~~~~~~~~~~~~~~~~~~~~~~~~~~~~~~~~~~~~~~~~~~~~~~~~~~~~~~~*/

#include <math.h>
#include <stdio.h>
#include "prototypes.h"

#define A 16807 
#define M 2147483647 
#define Q 127773
#define R 12614

double unif(long *X)
{
     long k;
     double unif;

     k = (*X)/Q;                   /* [x/q]                    */
     *X = A * (*X - k*Q) - R*k;    /* x - k*q = x mod q        */
     if(*X < 0) *X += M;
     unif = (*X)*((double)1.0/M);  /* make it uniform on (0,1) */
     return(unif);
}
\end{verbatim}
}
\subsection{makefile for rand}
{\tt
\begin{verbatim}
# Makefile for rand

HOME = /home/chip
HOMEBASE = $(HOME)/classes/s243/programs/hw2
HOMELIB = $(HOME)/lib
INCL = -I$(HOMELIB)/include
CC = gcc

rand: rand.o normal.o uniform.o moment.o $(HOMELIB)/util_lib.a 
        $(CC) -o rand -lm rand.o normal.o uniform.o moment.o $(HOMELIB)/util_lib.a 

rand.o: rand.c
        $(CC) $(INCL) -c rand.c 

normal.o: normal.c
        $(CC) $(INCL) -c normal.c 

uniform.o: uniform.c
        $(CC) $(INCL) -c uniform.c 

unif.o: unif.c
        $(CC) $(INCL) -c unif.c 

moment.o: moment.c
        $(CC) -c moment.c
\end{verbatim}
}
\subsection{output file from rand}
\label{sec:randout}
Running the {\tt rand} program ten times produced the ten samples
below.  A sample size of 25 was specified each time.
{\tt
\begin{verbatim}

UAverage = 0.456250, UVariance = 0.069789

The normal(0,1) random variables are:

-0.245908  0.420663  -1.339026  -0.224088  2.711854  -1.386384  
-0.602924  0.550597  -1.717494  -0.196892  -0.737359  -0.072223  
-0.358752  0.774570  0.633406  -0.436363  -0.245075  1.764727  
0.176432  -0.960782  -0.308958  -1.033925  -0.576253  -1.469689  
-0.032555  -0.653885  

Average = -0.196496, Variance = 0.987588, min = -1.717494, max = 2.711854 


UAverage = 0.524961, UVariance = 0.083246

The normal(0,1) random variables are:

0.757996  -1.387067  0.139614  -0.974996  -0.298773  0.701019  
-0.027365  0.402372  -0.314423  1.978237  0.801608  -0.854846  
0.051573  0.909785  0.987436  -1.322042  -0.066928  -0.397302  
-0.828935  1.112062  0.230128  0.850592  -0.540374  0.136476  
-2.533336  1.666251  

Average = -0.019500, Variance = 0.947179, min = -2.533336, max = 1.978237 


UAverage = 0.535653, UVariance = 0.073212

The normal(0,1) random variables are:

1.039667  -0.212589  1.648895  1.182762  -2.099287  1.118224  
1.271166  -0.350663  1.373938  -1.715097  -1.730662  2.434986  
-1.329970  0.265719  -0.738482  1.443447  -0.200154  2.674323  
-0.460446  1.509911  1.414562  -1.026580  0.154165  0.308496  
-2.082391  0.012989  

Average = 0.235758, Variance = 1.944441, min = -2.099287, max = 2.674323 


UAverage = 0.561852, UVariance = 0.083079

The normal(0,1) random variables are:

-1.627013  -1.946721  0.641673  1.599386  -0.050435  0.872663  
-1.023078  -0.172720  -0.479716  -2.624229  -1.393816  -0.781216  
-1.428000  -0.175682  0.804378  0.006943  -0.579453  0.469167  
-1.037056  -1.589434  -1.849918  -0.973552  0.777605  -0.021101  
0.015179  0.779816  

Average = -0.502646, Variance = 1.102117, min = -2.624229, max = 1.599386 


UAverage = 0.471777, UVariance = 0.117916

The normal(0,1) random variables are:

-0.180602  -0.033266  -0.491674  -0.210818  -0.005621  0.935598  
-0.750740  -0.150154  -0.651331  -0.605196  -0.568524  0.256189  
-0.499665  -0.216110  0.530548  -0.654962  -0.407916  -0.499960  
0.056185  -0.187132  -0.081403  0.359179  0.241485  -0.254034  
0.882858  -0.725366  

Average = -0.127483, Variance = 0.210574, min = -0.750740, max = 0.935598 


UAverage = 0.468296, UVariance = 0.078411

The normal(0,1) random variables are:

-0.538085  2.315870  -0.006784  -1.196539  1.234952  -2.092003  
0.278209  1.853595  -0.987914  1.145709  0.997270  0.430581  
2.112981  -1.594997  0.162695  0.883180  0.671533  0.501754  
-0.508260  -0.821739  -1.440829  0.851804  -1.373836  1.153124  
0.367494  -1.014375  

Average = 0.175991, Variance = 1.449950, min = -2.092003, max = 2.315870 


UAverage = 0.515804, UVariance = 0.097365

The normal(0,1) random variables are:

1.799999  -0.392986  -0.565985  0.775871  -1.454413  2.320997  
-0.352454  -0.103099  -2.437839  -0.288213  0.565715  -0.725542  
-0.475247  0.863397  -0.197961  0.213789  0.202876  0.665955  
0.879525  0.339340  0.554495  -0.809796  0.180958  -0.802160  
-0.112587  1.081356  

Average = 0.025785, Variance = 0.953816, min = -2.437839, max = 2.320997 


UAverage = 0.489400, UVariance = 0.088722

The normal(0,1) random variables are:

-1.164689  2.125476  -1.087806  -1.627695  -0.148151  -0.585310  
0.078002  1.434552  -0.490166  0.254162  0.830430  0.382266  
0.550786  0.555514  -0.691805  -1.458980  -0.299309  -0.331736  
-1.084327  1.043996  -0.185769  1.058329  -1.386962  0.052203  
0.452991  -0.702211  

Average = -0.068960, Variance = 0.921992, min = -1.627695, max = 2.125476 


UAverage = 0.565793, UVariance = 0.078631

The normal(0,1) random variables are:

-0.164288  0.031684  1.785622  0.642314  1.490821  -1.436317  
-0.340308  -0.757504  1.372027  1.243159  -3.225931  -0.149886  
0.932021  0.349455  -2.775168  0.033131  0.042626  -0.016562  
0.354986  2.069033  1.913653  0.436602  0.502835  0.265024  
0.438090  -1.048888  

Average = 0.201485, Variance = 1.621517, min = -3.225931, max = 2.069033 


UAverage = 0.455411, UVariance = 0.081871

The normal(0,1) random variables are:

-0.177334  -0.208764  0.414832  -0.465136  -0.305812  0.502667  
-0.042717  0.714116  -0.293542  2.073023  -0.052186  -1.267175  
-0.788828  0.296157  0.881320  0.702763  -0.114213  -0.010150  
-0.443612  0.101403  0.648522  -1.565041  1.401154  -0.953936  
1.095878  -1.801031  

Average = 0.085736, Variance = 0.678109, min = -1.801031, max = 2.073023 

\end{verbatim}
}

\pagebreak
\begin{thebibliography}{10}
\bibitem{loukides} M. Loukides and A. Oram  {\sl Programming with GNU
Software}.  O'Reilly and Associates, Inc., Sebastopol, 1997.
\bibitem{P&M} Park, S., and Miller, K. {\sl Communications of
the ACM,}, vol. 31, pp. 1192-1201, 1988.
\bibitem{schrage} Schrage, L. {\sl ACM Transactions on 
Mathematical Software,} vol. 5, pp. 132-138, 1979.
\end{thebibliography}
\end{document}


