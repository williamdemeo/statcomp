%**start of header
\documentclass{article}
\newcommand{\pipe}{$|\:$}
\usepackage{fullpage}
\usepackage{theorem}
{\theorembodyfont{\rmfamily} \newtheorem{definition}{Definition}[section]}
%**end of header

\begin{document}
\title{Statistics 243: \emph{class notes}}
\author{William J. De Meo}
\date{9/26/97 }
\maketitle
\section{Algorithms for Mean \& Variance}
\begin{definition}
The sample mean, $\bar x$, of a sample of n numbers is
\[\bar x = \frac{1}{n}\sum_i x_i \]
\end{definition}

\begin{definition}
The sample variance, $s^2$, is 
\[s^2 = \frac{1}{n-1}\sum_i (x_i - \bar x)^2 \]
\end{definition}
These are not the implementations we would use on the computer.
\begin{eqnarray*}
s^2 & = & \frac{1}{n-1}\sum_i (x_i - \bar x)^2\\
& = & \frac{1}{n-1}\sum_i(x_i^2 - 2\bar x x_i + \bar{x}^2)\\
& = & \frac{1}{n-1}\left[\sum_i x_i^2 -2\bar x \sum_i x_i +n\bar{x}^2\right]\\
& = & \frac{1}{n-1}\left[\sum_i x_i^2 - n\bar{x}^2\right]
\end{eqnarray*}
However, if both $\sum x_i^2$ and $n\bar{x}^2$ are large and nearly equal,
we may lose all the significant digits using this computation.
We need to know when this will be a problem.  Let $Q$ be the numerator
of $s^2$. Since $\sum_i x_i^2$ is roughly equal to $n\bar{x}^2$, we have
$\sum_i x_i^2 / n\bar{x}^2$ is roughly equal to 1.
\[ \frac{Q + n\bar{x}^2}{n\bar{x}^2} \approx 1\]
\[ \frac{Q}{n\bar{x}^2} + 1 \approx 1 \]

If CV denotes std.dev/mean then $(CV)^2 +1 \approx 1$

When $\mbox{CV}^2$ gets so small that adding to 1 doesn't change 1, we have a problem.

\subsection{Method of Provisional Means (BMDP)}
At each stage, recenter the data by subtracting the current estimate
of the mean.
\begin{definition}
\[\bar{x}^{(k)}= \frac{1}{k}\sum_i x_i \]
\[S^{(k)} = \sum_i (x_i - \bar{x}^{(k)})^2 \]
\[\bar{x}^{(k)} = \bar{x}^{(k-1)} + \frac{1}{k}(x_k - \bar{x}^{(k-1)})\]
\[s^{(k)} = s^{(k-1)} + (x^k - \bar{x}^{(k-1)} \ldots \]
\end{definition}
\subsection{Method of Subtracting the 1st Observation}
Calculate:
\[\sum(x-x_1)\]
(n extra flops) and 
\[\sum(x-x_1)^2 \]
(n extra flops)\\
Then use the desk calculator algorithm.  To get the right mean, we need to
add $x_1$ back, but the variance is still right.\footnote{Find a data set 
where this approach fails.}

\section{Random Number Generation}
We often want to know the property of a statistic when assumptions are 
violated.

Next time we'll look at generation of uniform r.v.s, some other distributions,
and some tests of randomness.
\end{document}



