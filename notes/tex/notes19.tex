%**start of header
\documentclass{article}
\newcommand{\pipe}{$|\:$}
\usepackage{fullpage}
%**end of header

\begin{document}
\title{Statistics 243: \emph{class notes}}
\author{William J. De Meo}
\date{October 22, 1997 }
\maketitle
\section{More About the Debugger}
The following abreviations are useful when using the debugger:\\
\begin{verbatim}
d short 
D long
S char*
F double
\end{verbatim}
For example, to display the first 20 doubles in an array x, do:\\
{\tt
\begin{verbatim}
&x[0]/20F
call command
\end{verbatim}
}
You might want to write a subroutine which prints a matrix in a nice way.
Then you can call it from the debugger:\\
{\tt  
\begin{verbatim}
double **;
void printmat(double *x, long nr, long nc);
\end{verbatim}
}
In debugger, do:\\\\
{\tt  
(dbx) call printmat(x,5,4)}\\\\
Consider making a file called print.c where you put all your print routines.

In the debugger, you can assign a variable a value before resuming 
program operation:\\\\
{\tt
assign variable = value\\\\
}
You will get tired of typing cotinue, run, step, etc.  so there is 
an alias facility:\\\\
alias name ``string'', e.g.\\
{\tt
alias s step\\
alias n next\\
alias e print\\
}
store it in a file in your home directory called .dbxinit.

Another feature available is the source command: {\tt source filename}

Suppose you put a stop in main and a stop in sub1.  Then do:
{\tt
step\\
printx\\
12\\
cont\\
$\vdots$\\
printx\\
75938492\\
}
So memory got trashed somewhere around $vdots$.  Make an alias:\\\\

{\tt alias m ``step; print  x''}\\\\

and put a bunch of m's in your file.  The output will show you the
line where the value of x has changed.

\section{Elimination Techniques}
{\bf Gaussian Elimination}\\
Performs elementary row operations on a matrix containing the left and
right hand sides of a system of equations.\\\\
Adjust algorithm:\\\\
{\tt
\begin{verbatim}
for k=1 to p
   b=a_{kk} 
   for i=1 to 2p
      a_{ki} = a_{ki}/b  /* if there's a problem with this step, we'll know */
   end
   for i=1 to p and i\neq k
      b=a_{ik}
      for j=1 to 2p
         a_{ij} = a_{ij} - b*a_{kj}
      end
   end
end
\end{verbatim}}

We have been keeping track of more than we need.  We will see a more
efficient way in a minute.  First, consider the augmented matrix
\[\hat X = [X:y]_{n \times (p+1)} \]
\[\hat X^t\hat X =  \left[ \begin{array}{cc}
X^tX & X^ty \\
y^tX & y^ty \end{array} \right]\]

Adjust algorithm applied to the first p rows transforms this matrix to 
\[\left[ \begin{array}{cc}
I & (X^tX)^{-1}X^ty \\
0 & y^t(I-X(X^tX)^{-1}X)y \end{array} \right]\]

and the bottom right entry contains the residual sum of squares.  
The matrix is now,
\[\left[ \begin{array}{cc}
I & \hat \beta \\
0 & RSS \end{array} \right]\]

We modify it to become the sweep(k) algorithm.  Operates on one column at 
a time.  The first sweep produces the beta hat in the top right entry
that we would get from regressing Y on only the first variable $x_1$.
The cool thing is that if we sweep a few columns, say the first three,
then decide we don't want the second column, resweep the second column --
that removes the second columns effect.

\subsection{The sweep(k) algorithm}
{\tt
\begin{verbatim}
let d = a_{kk}
for i=1 to p
a_{ki} = a_{ki}/d
end
 for i=1 to p, i \neq k
 b = a_{ik}
  for j=1 to p
  a_{ij} = a_{ij} - b*a_{kj}
  end
 a_{ik} = -b/d /* key step: the (ik)th element of the inverse is -b/d */
 end
a_{kk}= 1/d /* key step */
\end{verbatim}
}
Now our sweep operator does 
\[
\left[\begin{array}{cc}
X^tX & X^tY\\
y^tX & y^ty \end{array} \right] \]
we get
\[\left[\begin{array}{cc}
(X^tX)^{-1} & (X^tX)^{-1}X^ty \\
-y^tX(X^tX)^{-1} & y^t(I- X(X^tX)^{-1}X)y
\end{array}\right]\]
\end{document} 





