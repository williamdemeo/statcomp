%**start of header
\documentclass{article}
\newcommand{\pipe}{$|\:$}
\usepackage{fullpage}
%**end of header

\begin{document}
\title{Statistics 243: \emph{class notes}}
\author{William J. De Meo}
\date{October 8, 1997 }
\maketitle

\section{Computations with Matrices}
The operation $X^tX$ is performed as \\
{\tt t(X) \%*\% X }
\\
in S.  We will see how to do it faster.

\subsection{Addition and Subtraction}
Two matrices are conformable with repsect to addition and subtraction,
if they have the same number of rows and columns.  If $A$ and $B$ are 
conformable, then \\\\
\[C = A \pm B \Rightarrow C(i,j) = A(i,j) \pm B(i,j) \]
as in the following {\tt C} code segment:
{\tt
\begin{verbatim}
double *a, *b, *c;
for (i=0;i<nrow *ncol; i++) c[i] = a[i]+b[i];
cnow=c;
for(i=0;i<nrow*ncol;i++)
*(c++) = a[i] + b[i]
\end{verbatim}}
Here's a skeleton for a matrix add function:
{\tt
\begin{verbatim}
void madd(double *a, double *b, double *c, long n, long m){
    n*=m;
    while()
        (n--)*(c++) = *(a++) + *(b++);
}
\end{verbatim}}

\subsection{Multiplication}
Two matrices are conformable with respect to multiplication if their
inner dimensions are the same; ie and m by n matrix is conformable
with an n by p matrix.
\[
C = A * B \Rightarrow C_{ij} = \sum_{k=1}^n A_{ik} B_{kj}
\]
The dot product:
\[ 
(x,y) = \sum_{i=1}^n x_i y_j \]
The i,j the element of C is the dot product of the ith row of A
and the jth column of B.   For example:
{\tt
\begin{verbatim}
dot product for c[i][j]:
sum=0;
for(k=0;k<n;k++) 
    sum+=a[i][k]*b[k][j];

1  6 11    1 4
2  7 12    2 5
3  8 13    3 6
4  9 14
5 10 15
\end{verbatim}}
Suppose all matrices are stored in column major format in a one
dimensional array.  Then we use a \emph{stride} variable to indicate
how many elements you must advance to get to the next 
entry in the same row.  To change a col to a row or row to col (e.g
when using the transpose) just need to change the stride from nrows
to 1.  So, the dot product routine would look like:
{\tt
\begin{verbatim}
double dots(double *x, double *y, long n, long ix, long iy){
/* x is the address of x[i][0], y the address of y[0][j] */
/* ix and iy are the strides  */
/* if ix = iy = 1, we have the ordinary dot product */

double sum = 0;
for(i=0;i<n; x+=ix, y+=iy, i++) sum += *x * *y;
return sum;
\end{verbatim}}
So for matrices the stride of the first matrix should be nrows,
while the stride for second matrix is 1.  But if you want
to multiply the transpose of the first matrix and the second matrix
we put the stride of the first matrix to 1.  So we only need one
routine to do AB A'B and AB'.
\end{document}



