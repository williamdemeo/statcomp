%**start of header
\documentclass{article}
\usepackage{fullpage}
\newcommand{\pipe}{$|\:$}
%**end of header

\begin{document}
\title{Statistics 243: \emph{class notes}}
\author{William J. De Meo}
\date{August 29, 1997}
\maketitle

{\bf Topics}\\
\begin{enumerate}
\item Shell Services
\begin{itemize}
\item ~\ref{sec:redirection} Redirection
\item ~\ref{sec:jobcontrol} Job Control
\item ~\ref{sec:pathnames} Path Names
\end{itemize}
\end{enumerate}

\section{Shell Services}

We begin with some useful examples.  The command \\
{\tt head -12 \emph{filename}}\\
will show you the first 12 lines of the file called \emph{filename}. 
The command \\
{\tt tail -14 \emph{filename}}\\
will show you the last 14 lines of the file called  \emph{filename}.  The
command {\tt grep} finds regular expressions in its input.  For example,\\
{\tt grep \emph{string} \emph{filename} } \\
displays those lines of \emph{filename} containing occurances of \emph{string}.

\subsection{Redirection}
\label{sec:redirection}
To use {\tt grep} to find any errors in the output of \emph{program},
we redirect the program's output, making it input to the {\tt grep}
command.  To do this, run \emph{program} as follows: \\
{\tt \emph{program} \pipe grep -i error}\\
To learn about the kind of expressions {\tt grep} will accept, check 
out the article on regular expressions on Phil Spector's web page.\\
Some more examples of redirection:\\
{\tt > \&} redirects both stderr and stdout to a file \\
{\tt > > \&} appends both to file \\
{\tt \pipe \&}  pipes both to a file \\
To seperate stdout and stderr you could write \\
{\tt(\emph{command} >  outfile) > \& errorfile} \\
Once you direct stdout to {\tt outfile}, all that's left is stderr 
which you can redirect into {\tt errorfile}.

Suppose you want to execute a command on a bunch of files whose names are
all listed in the file filenames.  For example, suppose you want to 
copy all files in filenames to {\tt someotherdirectory}.  Use the ` charater: \\
{\tt cp `cat filenames` someotherdirectory}\\
Another example: \\
{\tt emacs `ls -t \pipe head -5`} \\
This opens emacs with 5 buffers containing the last 5 files.

\subsection{Job Control}
\label{sec:jobcontrol}
{\tt ps} lists processes you've initiated in a given shell.\\
If you run command on some shell on machine bugaboo, then run
{\tt ps} in a different shell window on bugaboo, it won't show you the job
command running on bugaboo.\\
{\tt ps -af} will list all processes on bugaboo. \\
{\tt ps -af \pipe grep \emph{username}} will list all processes on bugaboo which have
been initiated by username. \\
If you run a job that will run for a while and that you will check
periodically, you must remember on what machine you started the job.

In an interactive program \\
{\tt C-c} \\
interrupts\footnote{Notation: C-c means that you must hold down the Control key
  and press the letter c}. 
Some programs don't listen to this so the next thing you could do is\\ 
{\tt C-$\backslash$}  \\
which submits a quit signal. Many programs won't quit at optimal times
since this is an order to quit immediately.  The suspend signal\\
{\tt C-z}\\
puts job in the backround. 
To run any job in the backround, put an \& at the end of the command.  This does not
put output in backround, so any output will still be printed to terminal screen.
When you put a job in the backround you should, almost invariably, redirect stdout.
If you will remain at the terminal use \\
{\tt command >  output \& }\\
If you will leave the terminal, you have to also redirect stderror with \\
{\tt command > \& output \&}\\
You could also redirect output to other shells.  Type tty to find out what filename
corresponds to a given shell, then redirect output to that filename.  Remember not to kill
the shell!

If a job is running for a while and you want to leave, use {\tt C-z} and you get\\
{\tt [2] command (stopped)} \\
Now type {\tt bg} to get the job to start again in background.  To get it back into the foreground
type {\tt fg}.  If there are other jobs running, be more specific by
typing {\tt fg 2}, where 2 is the number of the job.  Job numbers can
be found by typing {\tt jobs}.

If you logout and then want to kill your job, get back on machine on which job
started.  Type {\tt ps -af \pipe grep \emph{ username}}.  
Note the process id (pid), say 27531, then type \\
{\tt kill 27531} \\
Easiest way to check that it worked is to type it again \\
{\tt !!} \\
If it's still running, you need something stronger.  Try \\
{\tt kill -9 27531} \\
If it still doesn't stop, it's a zombie, and the only way to stop it is to shut down the machine.
 
Miscellaneous note: \\
If you want a filename but don't really want a file you could use /dev/null
and nothing happens.

\subsection{Path Names}
\label{sec:pathnames}
When Phil puts samples in the s243 directory, it will be in the ta's directory.  That is, \\
{\tt /class/g/s243/s243/samples} \\
If you want to copy file {\tt test.c} from this directory you would write \\
{\tt cp /clas/g/s243/s243/samples} \\
But it's a pain to do this every time, so we need some shortcuts.\\
. is the current directory \\
.. is one level up \\
So from your home directory for the class account, you can type \\
{\tt cp ../s243/samples/test.c} \\
Another Example: 
What if you created a program in your home directory, but then do some clean up work
and want to move it into it's own directory.  Enter the following commands:
{\tt mkdir hw1} \\
{\tt cd hw1} \\
{\tt mv ../program.c }\\
Your home directory is denoted $\sim$, which is a shell service.

\end{document}
