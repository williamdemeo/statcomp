%**start of header
\documentclass{article}
\newcommand{\pipe}{$|\:$}
\usepackage{fullpage}

%**end of header

\begin{document}
\title{Statistics 243: \emph{class notes}}
\author{William J. De Meo}
\date{September 10, 1997}
\maketitle

{\bf Topics}\\
\begin{enumerate}
\item Operators
\begin{itemize}
\item ~\ref{sec:relational} Relational and Logical Operators
\item ~\ref{sec:increment} Increment and Decrement Operators
\item ~\ref{sec:assignment} Assignment Operators
\end{itemize}
\item Promotion

\end{enumerate}

\section{Operators}
\subsection{Relational and Logical Operators (cont.)}
\label{sec:relational}

Instead of \\\\
{\tt if(n>100)\{\\
\indent if(x = getdat())\{}
\\\\
one should consider using \\\\
{\tt if(n > 100 \&\& x=getdat()) }\\\\
The reason: one doesn't want to have to read many if statements 
to understand the structure of the program.\\\\
{\bf Other Remarks:}\\
The unary operator ! changes 0 to 1 and any nonzero to 0.\\
The assignment expression n=5 sets n equal to 5 but also evaluates to 5.  So 
the line\\\\
{\tt j=(n=5)}\\\\
assigns n equal to 5 and j equal to 5.\\\\
{\tt if(copy=n)\{ /* single equals ok */\}}\\\\
copies value of n to variable copy, then evaluates the if statements
if n is nonzero.  Comment is useful here since you might be tempted 
to correct with double equals.

\subsection{Increment and Decrement Operators}
\label{sec:increment}
{\bf Some Examples}\\
The listing \\\\
{\tt n=5;\\
x=n++;}\\\\
will result in {\tt x = 5} and {\tt n = 6}.  
On the other hand, the listing \\\\
{\tt n=5;\\
x=++5;}\\\\
results in {\tt x = 6} and {\tt n = 6}.

\subsection{Assignment Operators}
\label{sec:assignment}
{\bf Some Examples}\\
The listing\\\\
{\tt b = b * scale;}\\\\
is the same as\\\\
{\tt b *= scale;}
\\\\
and {\tt i++} is the same as {\tt i+=1} is the same as {\tt i = i+1}.

\subsection{Tertiary Operator}
\label{sec:tertiary}
{\bf Some Examples}\\\\
The listing\\\\
{\tt if(x > 0) ess = x;\\
else ess = 0;}\\\\
is the same as \\\\
{\tt ess = x > 0 ? x : 0 }\\\\
Another commonly used example:\\\\
{\tt min = x > y ? y : x;}

\section{Promotion}

The compiler will convert one value to a more meaningful value (more
bits) when it encounters them in an assignment; e.g.\\
short -> long \\
float -> double \\
The compiler will also truncate; e.g.\\\\
{\tt double x;\\
int ix;\\
x = 7.9; ix = x;}\\\\
yields {\tt ix = 7;}\\

To convert a character representation of an integer to integer or
float use:  {\tt atoi()} or {\tt atof()}

\end{document}

