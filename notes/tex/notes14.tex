%**start of header
\documentclass{article}
\newcommand{\pipe}{$|\:$}
\usepackage{fullpage}
%**end of header

\begin{document}
\title{Statistics 243: \emph{class notes}}
\author{William J. De Meo}
\date{9/26/97 }
\maketitle
\section{Matrix Multiplication}
The i,jth element is the dot product of the ith row of the first matrix with
the jth col of the second matrix.
To matrix multiply A and B, nrowa is the stride for matrix A, 
while 1 is the stride for B.  So the i,jth element is\\\\
{\tt dots(a+i, b+(j*nrowb),ncola,nrowa,1)}\\\\
Recall \\\\
{\tt dots(double *a, double *b, long n, long ix, long iy) \\
= dots(A, B, inner dimension, stride for A, stride for B)}\\
\\
To do the matrix multiplication $A^tB$, the ijth element is \\
\\
{\tt dots(a+(i*nrowa), b+(j*nrowb),nrowa, 1,1)}
\\\\
Now consider the problem of multiplying $X^tX$.
\begin{eqnarray*}
(X^tX)_{ij} & =& \sum_{k=1}^n(X^t)_{ik}X_{kj}\\
&  = &\sum_{k=1}^nX_{ki}X_{kj}\\
\end{eqnarray*}
The last line is the dot product of the ith col and the jth col.
{\tt
\begin{verbatim}
for(i=0;i<p;i++)
  for(j=0;j<=i,j++)
\end{verbatim}}
\indent $X^tX_{ij}$ = dot product of (col i of X and col j of X)

\end{document}



