%**start of header
\documentclass{article}
\newcommand{\pipe}{$|\:$}
\usepackage{fullpage}
%**end of header

\begin{document}
\title{Statistics 243: \emph{class notes}}
\author{William J. De Meo}
\date{October 29, 1997 }
\maketitle
\section{S}
A data frame allows you to think of columns as named variables and the rows
as observations.  It requires that all columns be the same length.  A list
is a completely general way of storing information.
\subsection{Getting Data Into S}
Assignment Operator $\leftarrow$ \\
x $\leftarrow$ 5 \\
x $\leftarrow$ c(2,3,4)\\
If you want to add a 20 to the end of x,\\
x $\leftarrow$ c(x,20)\\
x $\leftarrow$ c('hey', 'you')\\\\
The scan() function\\
x $\leftarrow$ scan()\\
1 10 12:15\\
17 18\\
\\
Typing a blank line and then return will tell S that you are done.\\
Suppose the file filename contains \\
1 2 3 4 \\
5 6 7 8\\
You  could read this into a matrix with\\
x $\leftarrow$ scan(``filename'')\\
x $\leftarrow$ matrix(x,nrow=2, ncol=4)\\ 
Which would give you \\
1 3 5 7\\
2 4 6 8\\
So we should instead do\\
x $\leftarrow$ matrix(x, nrow=2, ncol=4, byrow=T)\\
Instead we could compress the two steps into one with\\
x $\leftarrow$ matrix(scan(``filename''),nr=2,nc=4,byr=T)\\\\
The read.table function\\
x $\leftarrow$ read.table(``filename'',header=T)\\\\
When invoking Splus with the -e option, you should have the environmental
variable set to your editor preference:\\
setenv SEDITOR emacs \\\\
all data $\leftarrow$list(x,y,z)\\
x, y and z can be any kind of argument.\\\\
\subsection{Subscripts}
vectors\\
empty subscript x[]\\
EX:\\
x $\leftarrow$ matrix(0, 5, 5) is a five by five matrix with all entries 0.\\
To make a matrix of identical columns you would write:\\
y $\leftarrow$ matrix(c(1,5,7,9,11), 5, 5)\\
Suppose now we want to change the entries of x to all ones:\\
x $\leftarrow$ 1 would just set x to the scalar 1.\\
x[] $\leftarrow$ 1 puts all the elements of the 5 by 5 matrix to 1.\\
0 subscript is ignored.\\
positive numeric subscripts give you what you would expect.\\
negative subscripts give you the matrix with that subscript value removed.\\
EX:\\
x $\leftarrow$ c(1,3,5,7)\\
x[-2] would be 1,5,7\\
logical subscripts T or F\\
EX:\\
x $\leftarrow$ 1:10  gives x the values 1,2,3,...,10.\\
x < 5 will give you \\
T T T T F F F F F F  so you can use\\
x[x > 5] to extract from x the first four entries.







\end{document}



