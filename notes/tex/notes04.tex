%**start of header
\documentclass{article}
\newcommand{\pipe}{$|\:$}
\usepackage{fullpage}
%**end of header

\begin{document}
\title{Statistics 243: \emph{class notes}}
\author{William J. De Meo}
\date{September 12, 1997}
\maketitle

{\bf Topics}
\begin{enumerate}
\item C Programming
\begin{itemize}
\item ~\ref{sec:loops} Loops
\item ~\ref{sec:switch} The {\tt switch} Statement
\item ~\ref{sec:flow} Flow of Control
\item ~\ref{sec:running} Running the Program
\end{itemize}
\end{enumerate}



\section{C Programming}
{\bf TIP:}  If you want to comment out a bunch of code while debugging, but that
bunch of code has comments within it, surround the block with an {\tt if}
statement that will never be executed, e.g. {\tt if(0)}.

\subsection{Loops}
\label{sec:loops}

A very common mistake is to put a semicolon immediately after the 
while expression; e.g. \\
{\tt while(n $<$ 100); n++;} \\
creates an infinite loop.
\\

The {\it do while} loop:
\begin{tabbing}
doxxx\=doxxx\= statements\kill
\>{\tt do}\> \\
\>\>{\it statements} \\
\>{\tt while(\emph{expression});} \>
\end{tabbing}
Remember the semicolon after \emph{this} while expression!
\\
The listing
\begin{tabbing}
for( \= for( \= statements \kill
\>{\tt for(expr1; expr2; expr3)}\>\\
\>\>\emph{statements}
\end{tabbing}
Is the same as
\begin{tabbing}
for(\=for(\=statements\kill
\>{\tt expr1;}\>\\
\>{\tt while(expr2)\{}\>\\
\>\>\emph{statements}\\
\>\>expr3;\\
\>\}\>
\end{tabbing}

A good use for {\tt for} loops is array and vector processing:\\
{\tt for(sum = 0,i = 0; i $<$ n \&\& sum $<$ 1000; sum += x[i++]);}\\

\subsection{The {\tt switch} Statement}
\label{sec:switch}
{\tt
\begin{tabbing}
char\=switch\=switch\= \kill
char opt;\>\>\>\\
\>switch(opt)\>\>\\
\>\>case 'a':\>\\
\>\>\>\emph{statements}\\
\>\>\>break;\\
\>\>case 'b':\>\\
\>\>\>\emph{statements}\\
\>\>\>break;\\
\>\>default:\>\\
\>\>\>\emph{statements}\\
\>\>\>break;
\end{tabbing}
}
\subsection{Flow of Control}
\label{sec:flow}

{\tt break} provides an early exit from {\tt for, while, do while, and
switch}.  It does not break you out of an if statement.  You would need
a {\tt goto} statement for that.

{\tt continue} causes the next iteration of a {\tt while} or {\tt do while}
loop.  In a {\tt for } loop, it executes the third expression, and 
looping continues.

\subsection{Getting the Program to Run}
\label{sec:running}

An \emph{object file} is a file containing machine instructions usually
produced by a compiler.  In contains statements involving function calls,
but the function calls are not always in that object file.  So you must
link your object files, to create an \emph{executable file}.
An executable file contains machine instructions including the associated
functions.  It is a file that can actually be run.  The \emph{linker}
combines object files, searches libraries and creates an executable.
You often need to tell the linker which libraries you will use:\\
{\tt cc prog.c}\\
produces a file called {\tt a.out}.  Don't let it call it that.  Use instead:\\
{\tt cc -o prog prog.c}\\
To use the math library, use:\\
{\tt cc -o prog prog.c -lm}\\

Often you will want to compile different parts of your program into 
different object files called, say, {\tt part1.o, part1.o}, etc.
To compile a program without linking, use:\\
{\tt cc -c part1.c}\\
Once {\tt part1.c} works, you compile {\tt part2.c} with:\\
{\tt cc -o prog part1.o part2.c -lm}\\

\end{document}