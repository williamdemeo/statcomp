%**start of header
\documentclass{article}
\usepackage{fullpage}
\newcommand{\pipe}{$|\:$}
%**end of header

\begin{document}
\title{Statistics 243: \emph{class notes}}
\author{William J. De Meo}
\date{September, 1997}
\maketitle

{\bf Topics}\\
\begin{enumerate}
\item Some Basics of C Programming
\begin{itemize}
\item ~\ref{sec:gettingstarted} Getting Started \\
\item ~\ref{sec:declarations} Declarations \\
\item ~\ref{sec:compilers} C Compilers at the SCF
\end{itemize}
\end{enumerate}

\section{Some Basics of C Programming}
\subsection{Getting Started}
\label{sec:gettingstarted}

Some good places to look for information about the C programming language are:
\begin{itemize}
\item \emph{The C Programming Language 2nd Edition} (ANSI standard), 
Kernigan and Ritchie \\
\item Phil Spector's C programming tutorial on his web page.
\item FAQ in comp.lang.c at http://wwwcis.ohio-state.edu/hypertext/faq/usenet
\end{itemize}

Programs in C consist of one or more functions.  
Exactly one of the functions must have the name {\bf main}.
The simplest C program: main(){ }.
A more useful program is:
{\tt
\begin{tabbing}
\=main()\{\=scanf(``\%lf'',\&x);\=   \kill \\
\> main()\{\>\> \\
\>\>        double x,y;\> \\
\>\>        printf(``Enter first number: '');\> \\
\>\>        scanf(``\%lf'',\&x); /* read a long float (double) and */\>\\
\>\>\>                           /* put it at the address where x is */\\
\>\>        printf(``Enter second number: '');\> \\
\>\>        scanf(``\%lf'',\&y);\> \\
\>\>        printf(``The sum is \%lf $\backslash$n'', x+y);\> \\
\}\>\>\>\\
\end{tabbing}
}

\subsection{Declarations}
\label{sec:declarations}

{\bf char} is a character variable (1 byte integer)\\
{\bf int} is an integer (defaults to the natural size of the machine)
\footnote{On the Sun, the integer is very big so portability problems}\\
{\bf short} is an integer which specifies that it will be a small integer\\
{\bf long} is an integer which specifies that it will be as large as possible\\
long and short are the same on the Sun.  Use long for portability.\\
{\bf float} is a single precision floating point number\\
{\bf double} is a double precision floating point number\\
{\bf void} is a special declaration for functions that don't return anything

You can modify any of these data types by putting {\bf unsigned} in 
front of them for nonnegative numbers.  
It frees the sign bit, so your capacity is essentially doubled.

\subsection{C Compilers at the SCF}
\label{sec:compilers}

Most of the computing at the SCF is now done on Sun UltraSparc work
stations.  These work stations run the Sun Solaris operating system.
On this system, there are generally two flavors of C compiler: cc and gcc.  
cc was developed by Sun so it is the most efficient when run on a Sun workstation.

\end{document}
        





