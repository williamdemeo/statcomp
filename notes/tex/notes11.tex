%**start of header
\documentclass{article}
\newcommand{\pipe}{$|\:$}
\usepackage{fullpage}
%**end of header

\begin{document}
\title{Statistics 243: \emph{class notes}}
\author{William J. De Meo}
\date{10/1/97 }
\maketitle
\section{Uniform Random Number Generator}
1. Linear Congruential U(0,1) Generator
This generator is based on four numbers
\begin{itemize}
\item m = modulus 
\item c = increment
\item a = multiplier
\item $X_0$ = seed
\end{itemize}

The formula for this generator is \\
\[
X_{n+1} = \mbox{mod}(aX_n + c, m)
\]
\\
For $X_0$ system clock value is one possible source of a starting value. 
We can set $X_0$ to the clock value with the line:\\\\
{\tt seed = time();}

If c = 0, the method is called a \emph{multiplicative congruential generator}.

By their very nature, these generators have a cycle, or period 
which is the number of unique vlues which are produced 
before it starts repeating.  We want a period equal to the number
of unique values in the computer.  So we should pick the largest
unsigned integer for $m$, or $2^p$ where $p$ is the number of bits on the 
computer.  But this is the same as just using the formula
\[X_{n+1} = aX_n + c \]

It can be shown that a mixed congruential generator will have period m
iff 
\begin{enumerate}
\item c is relatively prime to m 
\item mod(a,p) = 1 for all prime factors p of m 
\item mod(a,4) = 1 if 4 is a factor of m 
\end{enumerate}

\subsection{Composite Generators}
\[X_{n+1} = \mbox{mod}(a_1 X_n + c_1 , m)\]
\[Y_{n+1} = \mbox{mod}(a_2 Y_n +c_2, m)\]
\[W_n = \mbox{mod}(X_n + Y_n, m)\]

\subsection{Quadratic Congruential Generator}
\[X_{n+1} = \mbox{mod}(a X_n^2 + aX_n + c, m) \]
 
\subsection{Additive Generators}
see {\tt man 3 random}
\[X_n \mbox{mod}(X_{n - r_1} + X_{n - r_2}, m), n \geq max(r_1, r_2) \]

To start, the first $max(r_1, r_2)$ numbers are chosen arbitrarilly
$r_1 = 24, r_2 = 55$ possibly good starting values.

\subsection{Feedback shift Register Techniques (Tausworthe generators)}
Linear recurrence relation among the bits of the random number.
\[ a_k = mod((c_p a_{k-p} +c_{p-1}a_{k-p+1} + \cdots + c_1 a_{k-1}, 2) \]

The $\{c_i\}$ are fixed and equal to 0 or 1.

\subsection{Shuffling}
We can make any random number generator more random by using \emph{shuffling}.
The procedure is as follows:
\begin{enumerate}
\item Initialization 
\begin{itemize}
\item Generaute an array of, say, 100 random numbers 
\item You should have it automatically initialized 
\end{itemize}
\item Generate another random number y to start the process. 
\item Each time you want a random number, use y to find an index
into v: {\tt index = (int)( 100 * y((double)m)}
where m is the modulus.
\item Set y = v[index] 
\item Replace v[index] with a new random number 
\item Return y.
\end{enumerate}


\end{document}





