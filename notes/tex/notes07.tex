%**start of header
\documentclass{article}
\newcommand{\pipe}{$|\:$}
\usepackage{fullpage}
%**end of header

\begin{document}
\title{Statistics 243: \emph{class notes}}
\author{William J. De Meo}
\date{9/22/97}
\maketitle
{\bf Topics}
\begin{enumerate}
\item Functions
\begin{itemize}
\item ~\ref{sec:communication} Communication Between Functions
\item ~\ref{sec:variables} External (Global) Variables
\end{itemize}
\end{enumerate}

\section{Functions}
A good function for summing:\\\\
{\tt double sum(double *x, long n)\\
\{\\ 
double xs;\\
xs=0;\\
while(n--) xs += *(x++);\\
return(xs);\\
\}}\\
\\
We pass the values stored at the address {\tt x}, and use these values
to compute the sum.  The function never changes the values stored at {\tt x}.

\subsection{Communication Between Functions}
\label{sec:communication}

The \emph{scope} of a variable describes how widely the value of that variable 
is known to your program.  Generally speaking, one should only pass to a function the
information that the function really needs.  This is a concept known
as information hiding, and is based on the idea that, the less a
function has access to, the less it can damage.  A function should 
rarely need access to anything not passed to it in its argument list.
There are exceptional cases, however:  
\begin{itemize}
\item Suppose we have a large graphics subsystem.
We have variables like scale, origin, line type, axis information, etc. that
all functions need to know.  We could put all these variables in a structure 
and pass it to every function.
\item Suppose we have a function which accepts another function as an argument, but
has a limited argument list which we don't want to change.  Then we need a
new way to communicate between functions. 
\end{itemize}
Keep in mind that these are special circumstances.  In general you should be 
wary of giving functions access to too much.

\subsection{External (Global) Variables}
\label{sec:variables}

In one and only one file, we declare an external or global variable outside of the
main function.\\\\
{\tt double *x;\\\\
main()\\
\{ $\ldots$}\\\\
This variable is automatically accessible by anything inside this file.  To access
that variable from another file, declare it with the {\tt extern} qualifyer, which
says, don't create new memory for this variable, go out and look for the existing
variable.  E.g., in the file funct1.c, we have:\\\\
{\tt
extern double *x;\\\\
funct1()\\
\{ $\ldots$}\\\\
In a third file, we might only want access to the global variable from within 
the {\tt funct3()} function:\\\\
{\tt
funct3()\\
\{\\
extern double *x;\\
$\ldots$}

In the first case scenario, that of a large graphics subsystem, it is a stupid idea
to declare all the variables as extern.  Instead, for communication within
a single file, declare the variable outside any curly braces, using the {\tt static} 
qualifier.  The variable is now available to any function in that file.  

Don't use these techniques just because defining the same kind of variable is getting
tiresome.  If something goes wrong with the value of a variable that has a large scope,
you will have to find out what's going on in each function that has access to the variable --
a big pain.

Another effect of the static qualifier is that the memory allocated for a static variable
is untouched throughout the program, so the value store there can only be changed by 
explicitly changing the static variable.  i.e. the compiler will not reallocate the memory.
Ex:\\\\
{\tt
double func(long int n)\\
\{\\
static double *x;\\
static long start=1;\\
if(start==1){\\
x = dmalloc(n);\\
start = 0;\\
}\\
$\ldots$}

\end{document}





